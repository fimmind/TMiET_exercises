\documentclass[a4paper, 12pt]{article}
\usepackage[a4paper]{geometry}

\usepackage[T1,T2A]{fontenc}
\usepackage[utf8]{inputenc}
\usepackage[english,russian]{babel}

\usepackage{amsmath}
\usepackage{amssymb}
\usepackage{amsthm}
\usepackage{mathrsfs}
\usepackage{mathtools}
\usepackage{booktabs}

\usepackage[
    colorlinks=true,
    allcolors=black,
    urlcolor=blue,
]{hyperref}

\renewcommand{\thesection}{}
\renewcommand{\thesubsection}{}
\renewcommand{\thesubsubsection}{}
\renewcommand{\theparagraph}{}
\setlength{\parindent}{0pt}

\theoremstyle{definition}
\newtheorem{exercise}{Упр.}

\begin{document}

\subsection{Преобразование Фурье, пространство Шварца} % 08.09.23

\begin{exercise} % p.7, ex. 1
    Доказать, что для любого многочлена \({ P_{m}(x) }\), \({ x \in \mathbb R^{n} }\), со значениями в \({ \mathbb R^{m} }\) функция \({ P_{m}(x) e^{-\lvert x \rvert^2} }\) лежит в пространстве Шварца \({ \mathcal S(\mathbb R^{n}) }\).
\end{exercise}

\begin{exercise}[Коммутационные соотношения]
    Показать, что для любой функции \({ u \in \mathcal S(\mathbb R^{n}) }\) и мультииндекса \({ \alpha \in \mathbb Z_{+}^{n} }\) выполняются равенства
    \[
        \begin{gathered}
            \mathcal F_{x \to \xi} \partial_{x}^{\alpha} u(x) = (i\xi)^{\alpha} \mathcal F_{x \to \xi} u(x)\,, \\
            \mathcal F_{x \to \xi} \left[ x^{\alpha} u(x) \right] = \left[ i \frac{\partial}{\partial \xi} \right]^{\alpha} \mathcal F_{x \to \xi}u(x)\,.
        \end{gathered}
    \]
\end{exercise}

\begin{exercise} % p.7, ex. 4
    Рассмотреть последовательность функций \({ \left\{ h_{p}(x) \right\} }\), \({ x \in \mathbb R^{1} }\), где \[
        h_{p}(x) = \begin{cases}
            p\,, & \lvert x \rvert < \frac{1}{2p}\,, \\
            0\,, & \lvert x \rvert > \frac{1}{2p}\,.
        \end{cases}
    \]

    Доказать, что последовательность \({ \left\{ h_{p}(x) \right\} }\) фундаментальна по норме пространства \({ H^{-1}(\mathbb R^{1}) }\) и что для всякой функции \({ \varphi(x) \in S(\mathbb R^{1}) }\) справедливо равенство \[
        \lim_{p \to \infty} \int_{\mathbb R^{1}} h_{p}(x) \varphi(x)\: dx = \varphi(0)\,.
    \]
\end{exercise}

\begin{exercise} % p.7, ex. 12
    Вычислить преобразование Фурье для следующих функций:
    \begin{enumerate}
        \item \({ \displaystyle f(x) = \frac{1}{\pi} \cdot \frac{\varepsilon}{x^2 + \varepsilon^2} \quad (\varepsilon > 0) }\);
        \item \({ \displaystyle f(x) = \sqrt{\frac{n}{4\pi}} e^{-n x^2 / 4} }\);
        \item \({ \displaystyle f(x) = \frac{1}{\pi} \frac{\sin nx}{x} }\).
    \end{enumerate}
\end{exercise}

\subsection{Пространства Соболева} % 15.09.23

\begin{exercise} % p. 9, ex. 13
    Определить для какого максимального \({ k }\) функция \({ \varphi(x) }\) принадлежит пространству \({ H^{k}(\mathbb R^{1}) }\), где \[
        \varphi(x) = \begin{cases}
            0\,, & x \leqslant 0 \text{ или } x \geqslant 2\,, \\
            x\,, & 0 \leqslant x \leqslant 1\,, \\
            2 - x\,, & 1 \leqslant x \leqslant 2\,.
        \end{cases}
    \]
\end{exercise}

\begin{exercise} % p. 10, ex. 23
    Дана функция \({ f(x) \in \mathcal S(\mathbb R^{1}) }\)  такая, что \({ \int_{\mathbb R^{1}} f(x)\: dx = 1 }\). Доказать, что последовательность \({ n \cdot f(nx) }\) при \({ n \to \infty }\) сходится в \({ H^{-1}(\mathbb R^{1}) }\) к \({ \delta(x) }\).
\end{exercise}

\begin{exercise} % p. 9, ex. 23
    Найти преобразование Фурье функции \({ f(x) = \frac{1}{x^2 - k^2 - i\varepsilon} }\), \({ x \in \mathbb R^{3} }\), \({ k \in \mathbb R^{1} }\), \({ \varepsilon > 0 }\).
\end{exercise}

\begin{exercise} % p. 10, ex. 25
    Обобщённая функция \({ \delta(\lvert x \rvert - a) }\), \({ x \in \mathbb R^{n} }\) из \({ H^{-[n/2]-1}(\mathbb R^{n}) }\) определяется с помощью функционала, задаваемого равенством \[
        \int_{\mathbb R^{n}}  \delta(\lvert x \rvert - a) \varphi(x)\: dx = \int_{\lvert x \rvert = a} \varphi(\xi)\: d\sigma_{\xi}\,,
    \] где \({ \varphi(x) \in H^{[n / 2]-1}(\mathbb R^{n}) }\), \({ d\sigma_{\xi} }\) - элемент поверхности сферы радиуса \({ a }\) в \({ \mathbb R^{n} }\) с центром в начале координат. Доказать ограниченность функционала задачи и найти преобразование Фурье функции \({ \delta(\lvert x \rvert - a) }\).
\end{exercise}

\begin{exercise} % p. 11, ex. 29
    Найти производную функции \({ \theta(x) e^{-\alpha x} }\), \({ \alpha > 0 }\), где \[
        \theta(x) = \begin{cases}
            1\,, & x > 0\,, \\
            0\,, & x < 0\,.
        \end{cases}
    \]
\end{exercise}

\begin{exercise} % p. 12, ex. 34
    Упростить выражения, вычислив входящие в них производные
    \footnote{При вычислении производных от функций \({ f(x) \in H^{k}(\mathbb R^{n}) }\) полезно использовать тождество \[
        \frac{\partial}{\partial x_j} f(x) = \mathcal F^{^{-1}}_{\xi \to x} (i\xi_{j} \mathcal F_{x \to \xi}f(x))\,.
    \]}:
    \begin{enumerate}
        \item \({ \displaystyle \frac{1}{4\pi} (- \Delta + 1) \frac{e^{-\lvert x \rvert}}{\lvert x \rvert} }\), \({ x \in \mathbb R^3 }\); \\
        \item \({ \displaystyle \left( \frac{\partial}{\partial t} - a^2 \Delta \right) \frac{\theta(t) e^{-\frac{x^2}{4t}}}{2a \sqrt{\pi t}} }\), \({ x, t \in \mathbb R^{1} }\); \\
        \item \({ \displaystyle (\Delta + k^2) \frac{e^{ik \lvert x \rvert}}{2ik} }\), \({ x \in \mathbb R^{1} }\);
        \item \({ \displaystyle \Delta\left( \frac{1}{\lvert x \rvert} \right) }\), \({ x \in \mathbb R^3 }\).
    \end{enumerate}
\end{exercise}

\begin{exercise}
    Пусть \({ u \in C^{\infty}(\mathbb R^{n}) }\) и \({ u(x) \equiv 0 }\) при \({ \lvert x \rvert > \frac{\pi}{2} }\). Доказать, что норма \[
        \lVert u \rVert'_{s} \overset{\text{def}}= \sqrt{\sum \lvert \hat{u}(k) \rvert^2 (1 + k^2)^{s}}\,,
    \] где \({ \hat{u}(k) }\) есть \({ k }\)-ый коэффициент разложения функции \({ u }\) в ряд Фурье \[
        u(x) = \sum \hat{u}(k) e^{ikx}\,,
    \] эквивалентна норме \({ \lVert u \rVert_{s} }\) пространства \({ H^{k}(\mathbb R^{n}) }\).
\end{exercise}

\subsection{Псевдодифференциальные операторы} % 29.09.23

\begin{exercise}
    Даны символы \({ a = f(x_1, x_2) \xi_1^2 + \xi_2^2 }\) и \({ b = \left( f(x_1, x_2) \xi_1^2 + \xi_2^2 \right)^{-1} }\). Вычислить \({ 3 }\) первых слагаемых\footnote{Имеются в виду первые слагаемые ряда, эквивалентого символу композиции операторов по теореме Кона-Ниренберга.} символа оператора \({ \operatorname{Op}(a) \operatorname{Op}(b) }\).
\end{exercise}

\begin{exercise}
    Дан символ \({ b(x, \xi) = \xi_1^2 + x_1^3 \xi_2^2 }\). Найти первые \({ 2 }\) слагаемых символа \({ a }\) такого, что \({ \operatorname{Op}(a)^2 = \operatorname{Op}(b) }\).
\end{exercise}

\subsection{Понятие гладкого многообразия} % 13.10.23

\begin{exercise}
    Показать, что две стереографические проекции единичной сферы в \({ \mathbb R^{3} }\) на плоскость, проходящую через центр этой сферы, определяют на ней гладкий атлас.
\end{exercise}

\subsection{Теория Фредгольма} % 03.10.23

\begin{exercise}
    Пусть \({ H }\) - комплексное гильбертово пространство и \({ x, y \in H }\).
    Обозначим через \({ (\widehat{x, y}) }\) угол между \({ x }\) и \({ y }\) относительно естественно индуцированного скалярного произведения в овеществлении пространства \({ H }\).
    Верно ли, что \({ \cos (\widehat{x, y}) = \frac{\operatorname{Re}\, (x, y)}{\lvert x \rvert \cdot \lvert y \rvert} }\)?
\end{exercise}

\begin{exercise}
    Пусть \({ \left\{ \lambda_j \right\}_{j=0}^{\infty} }\) - фиксированная последовательность комплексных чисел и оператор \({ A }\) в пространстве \({ \ell^2(\mathbb Z_{+}) }\) определяется формулой \[
        A(x_0, x_1, \ldots, x_n, \ldots) = (\lambda_0 x_0, \lambda_1 x_1, \ldots, \lambda_n x_n, \ldots)\,.
    \] Показать, что
    \begin{enumerate}
        \item \({ \lVert A \rVert = \sup_{j} \lvert \lambda_j \rvert }\);
        \item \({ A }\) --- оператор конечного ранга \({ \iff }\) последовательность \({ \left\{ \lambda_j \right\} }\) финитна;
        \item если \({ \lambda_j \mapsto 0 }\) при \({ j \to \infty }\), то \({ A }\) - компактный оператор.
    \end{enumerate}
\end{exercise}

\subsection{Дифференциальные формы в \({ \mathbb R^{n} }\)}

\begin{exercise}
    Пусть \({ V }\) --- конечномерное векторное пространство. Показать, что для любых \({ a \in \Lambda^{k}(V) }\) и \({ b \in \Lambda^{l}(V) }\) выполняется \[
        ab = ba \cdot (-1)^{kl}\,.
    \]
\end{exercise}

\begin{exercise}
    Пусть \({ \left\{ e_{j} \right\}_{j = 1}^{n} }\) - базис в линейном пространстве \({ V }\) и \({ a_i = \sum_j a_{ij} e_{j} }\) для \({ i \in \overline{1, n} }\). Показать, что \[
        a_1 \cdots a_n = k \cdot e_{1} \cdots e_n \in \Lambda^{n}(V)\,,
    \] где \({ k = \det (a_{ij}) }\).
\end{exercise}

\begin{exercise}
    Пусть на сфере \({ \mathbb S^2 \subseteq \mathbb R^3 }\) вне северного и южного полюсов выбраны стандартные сферические координаты \[
        \begin{gathered}
            x = \cos \varphi \sin \theta\,, \\
            y = \sin \varphi \sin \theta\,, \\
            z = \cos \theta\,.
        \end{gathered}
    \] Показать, что дифференциальная \({ 1 }\)-форма \({ \omega \coloneqq \sin \theta\: d\varphi\: d\theta }\), продолженная по непрерывности на полюса, гладка в каждой точке сферы.
\end{exercise}

\begin{exercise}
    Найти числа Бетти и Эйлерову характеристику для следующих многообразий:
    \begin{enumerate}
        \item \({ [0, 1] }\),
        \item \({ \mathbb S^1 }\),
        \item \({ [0, 1] \times [0, 1] }\),
        \item \({ \mathbb S^1 \times \mathbb S^1 }\).
    \end{enumerate}
\end{exercise}

\begin{exercise}
    Вычислить когомологии де Рама с компактным носителем для многообразия \({ M = \mathbb R }\).
\end{exercise}

\begin{exercise}
    Показать, что
    \begin{enumerate}
        \item сфера ориентируема;
        \item лист Мёбиуса не ориентируем.
    \end{enumerate}
\end{exercise}

\subsection{Расслоения, связности}

\begin{exercise}
    Пусть \({ \nabla = d + a }\) --- связность. Вычислить кривизну \({ \nabla^2 }\).
\end{exercise}

\begin{exercise}
    Пусть \({ S }\) и \({ N }\) - южный и северный полюса сферы \({ \mathbb S^2 }\) и \({ U_{+} \coloneqq \mathbb S^2 \setminus \left\{ S \right\} }\), \({ U_{-} \coloneqq \mathbb S^2 \setminus \left\{ N \right\} }\). Тогда для любого отображения \[
        A : U_+ \cap U_- \to \operatorname{Gl}(k, \mathbb C)
    \]
    имеем векторное расслоение \({ E = \left[ U_+ \times \mathbb C^{k} \right] \amalg \left[ U_- \times \mathbb C^{k} \right] }\), в котором отождествляются элементы \({ (x, v) \in U_+ \times \mathbb C^{k} }\) и \({ (x, A(x)v) \in U_- \times \mathbb C^{k} }\) для всех \({ x \in U_+ \cap U_- }\). Найти отображение \({ A(x) }\) такое, что \({ E \simeq T \mathbb S^2 }\), где \({ T\mathbb S^2 }\) есть касательное расслоение сферы с комплексной структурой, определяемой поворотом на \({ \frac{\pi}{2} }\).
\end{exercise}

\begin{exercise}
    \label{ex:connection_induced_by_projections}
    Пусть \({ E = \operatorname{Im} P }\), где \({ P : X \to \operatorname{Mat}(N, \mathbb C) }\) есть гладкое семейство проекторов\footnote{Матрица \({ A \in \operatorname{Mat}(N, \mathbb C) }\) называется проектором, если \({ A^2 = A }\).}. Доказать, что отображение \[
        \nabla = Pd : C^{\infty}(X, E) \to \Omega^{1}(X, E)
    \] есть связность в расслоении \({ E }\). Найти кривизну этой связности.
\end{exercise}

\begin{exercise}
    В условиях упражнения~\ref{ex:connection_induced_by_projections} вычислить кривизну связности \({ \nabla }\) для проектора Ботта \({ P : \mathbb C \to \operatorname{Mat}(2, \mathbb C) }\), определяемого соотношением \[
        P(z) = \frac{1}{1 + \lvert z \rvert^2} \begin{pmatrix}
            1 & z \\
            \bar z & \lvert z \rvert^2
        \end{pmatrix}\,.
    \] 
\end{exercise}

\end{document}
