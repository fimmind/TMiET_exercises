\documentclass[a4paper, 12pt]{article}
\usepackage[a4paper]{geometry}

\usepackage[T1,T2A]{fontenc}
\usepackage[utf8]{inputenc}
\usepackage[english,russian]{babel}

\usepackage{amsmath}
\usepackage{amssymb}
\usepackage{amsthm}
\usepackage{mathrsfs}
\usepackage{mathtools}
\usepackage{booktabs}

\usepackage[
    colorlinks=true,
    allcolors=black,
    urlcolor=blue,
]{hyperref}

\renewcommand{\thesection}{}
\renewcommand{\thesubsection}{}
\renewcommand{\thesubsubsection}{}
\renewcommand{\theparagraph}{}
\setlength{\parindent}{0pt}

\theoremstyle{definition}
\newtheorem{exercise}{Упражнение}

\usepackage{mhchem}
\usepackage[lastexercise]{exercise}

\begin{document}

\subsection{Преобразование Фурье, пространство Шварца} % 08.09.23

\begin{exercise} % p.7, ex. 1
    Доказать, что для любого многочлена \({ P_{m}(x) }\), \({ x \in \mathbb R^{n} }\), функция \({ P_{m}(x) e^{-\lVert x \rVert^2} }\) лежит в пространстве Шварца \({ \mathcal S(\mathbb R^{n}) }\).
\end{exercise}

\begin{exercise}[Коммутационные соотношения]
    Показать, что для любой функции \({ u \in \mathcal S(\mathbb R^{n}) }\) и мультииндекса \({ \alpha \in \mathbb Z_{+}^{n} }\) выполняются равенства
    \[
        \begin{gathered}
            \mathcal F_{x \to \xi} \partial_{x}^{\alpha} u(x) = (i\xi)^{\alpha} \mathcal F_{x \to \xi} u(x)\,, \\
            \mathcal F_{x \to \xi} \left[ x^{\alpha} u(x) \right] = \left[ i \frac{\partial}{\partial \xi} \right]^{\alpha} \mathcal F_{x \to \xi}u(x)\,.
        \end{gathered}
    \]
\end{exercise}

\begin{exercise} % p.7, ex. 4
    Рассмотреть последовательность функций \({ \left\{ h_{p}(x) \right\} }\), \({ x \in \mathbb R^{1} }\), где \[
        h_{p(x)} = \begin{cases}
            p\,, & \lvert x \rvert < \frac{1}{2p}\,, \\
            0\,, & \lvert x \rvert > \frac{1}{2p}\,.
        \end{cases}
    \]

    Доказать, что последовательность \({ \left\{ h_{p}(x) \right\} }\) фундаментальна по норме пространства \({ H_{-1}(\mathbb R^{1}) }\) и что для всякой функции \({ \varphi(x) \in S(\mathbb R^{1}) }\) справедливо равенство \[
        \lim_{p \to \infty} \int_{\mathbb R^{1}} h_{p}(x) \varphi(x)\: dx = \varphi(0)\,.
    \]
\end{exercise}

\begin{exercise} % p.7, ex. 12
    Вычислить преобразование Фурье для следующих функций:
    \begin{enumerate}
        \item \({ \displaystyle f(x) = \frac{1}{\pi} \cdot \frac{\varepsilon}{x^2 + \varepsilon^2} \quad (\varepsilon > 0) }\);
        \item TODO
    \end{enumerate}
\end{exercise}

\end{document}
